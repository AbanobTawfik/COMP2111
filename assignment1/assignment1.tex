\documentclass{article}
\usepackage[utf8]{inputenc}
\usepackage[a4paper, total={6in, 8in}]{geometry}
\usepackage{graphicx}
\newcommand*\moveToRight[1]{\hspace*{0em plus 1fill}\makebox{(#1)}}
\newcommand*\fixindent{ \hspace{1pt}\\}
%this command below is not my work was used for quality of life
%link to original post 
%https://tex.stackexchange.com/questions/330588/how-to-produce-given-number-of-quad-in-math
\newcommand{\myquad}[1][1]{\hspace*{#1em}\ignorespaces}
\title{Assignment One}
\author{Abanob Tawfik\\z5075490}
\date{March 2019}

\begin{document}

\maketitle
\section{Problem 1}
Given two sets A and B, we define the operation ∗ as follows:
\begin{center} A * B := $(A \cap B)^c$ \end{center} \\
Answer the following questions using the Laws of Set Operations (and any related results proven in lectures or tutorials) to justify your answer:\\\\

\fixindent{}
(a) What is  (A * B) * (A * B)?\moveToRight{5 marks}\\\\Solution:\\
from using the supplied definition of the * operator\\\\
\hspace*{30pt}(A * B) * (A * B) = $(A \cap B)^c$ * $(A \cap B)^c$ \moveToRight{Supplied definition of *}\\
\moveToRight{Substituting A * B}\\
\hspace*{110pt} = ($(A \cap B)^c \cap(A \cap B)^c)^c$ \moveToRight{using definition of *}\\
\\ Let's say set C = $(A \cap B)^c$\\\\
\hspace*{110pt} = $( C \cap C)^c$ \moveToRight{using our substitution C = $(A \cap B)^c$}\\
\hspace*{110pt} = $C^c$ \moveToRight{using law of idempotence}\\
\hspace*{110pt} = $((A \cap  B)^c)^c$ \moveToRight{substituting C}\\
\hspace*{110pt} = $A \cap B$ \moveToRight{using double complementation}\\\\\\
So finally, $(A * B) * (A * B) = A \cap B$\\
\newpage

\fixindent{} 
(b) Express $A^c$ using only A, * and parentheses (if necessary).\moveToRight{5 marks}\\\\Solution:\\
one way to express $A^c$ is $(A \cup A)^c$ OR $(A \cap A)^c$ \moveToRight{law of idempotence}\\
using this fact, i will make the claim $A^c$ = $ A * A$\\\\
\hspace*{30pt}A * A = $(A \cap A)^c$ \moveToRight{Supplied definition of *}\\
\hspace*{57pt} = $(A)^c$ \moveToRight{law of idempotence}\\
\hspace*{57pt} = $A^c$ \moveToRight{removing parenthesis}\\
So finally, A * A  = $A^c$, using only A, * and parenthesis.\\\\\\\\

\fixindent{}
(c) Express $A \cup B$ using only $A, B, *$ and parenthesis (if necessary).\moveToRight{10 marks}\\\\Solution:\\
One way to express $A \cup B$ is $(A^c \cap B^c)^c$ \moveToRight{de Morgan's Law and double complementation}\\
using this fact, i will make the claim $A \cup B =  (A * A) * (B * B)$\\\\
\hspace*{30pt}$(A * A) * (B * B) = ((A \cap A)^c \cap (B \cap B)^c)$ \moveToRight{Supplied definition of *}\\
\hspace*{110pt} = $((A)^c \cap (B)^c)^c$ \moveToRight{law of idempotence}\\
\hspace*{110pt} = $(A^c \cap B^c)^c$ \moveToRight{removing parenthesis}\\
\hspace*{110pt} = $(A^c)^c \cup (B^c)^c$ \moveToRight{de Morgan's Law}\\
\hspace*{110pt} = $A \cup B$ \moveToRight{using double complementation}\\
So finally, $(A * A) * (B * B)  = A \cup B$, using only $A, B, *$ and parenthesis.\\\\\\\\

\newpage
\section{Problem 2}
A binary tree is a data structure where each node is linked to at most two successor nodes:\\\\
\begin{centering}\includegraphics{tree.png}\end{centering}

\fixindent{}
If we allow empty binary trees (trees with no nodes), then we can simplify the possibilities of zero, one, or two successors by saying a node has exactly two children which are binary trees\\\\

\fixindent{}
(a) Give a recursive definition of the binary tree data structure. Your definition may be concrete (i.e. code)
or abstract, as long as it is clear what the base and recursive cases are. \moveToRight{4 marks}\\\\Solution in java:\\
\hspace*{150pt}public class Tree\textless  data\textgreater\\
\hspace*{150pt}\{\\
\hspace*{165pt} public data D;\\
\hspace*{165pt} Tree Left;\\
\hspace*{165pt} Tree Right;\\
\hspace*{150pt}\}\\\\
For this code, a base case would be \\
Tree $==$ NULL\\
The recursive case would be to pass in the attributes that are\\
Left and Right,which are also trees.\\
Leaves occur when Left and Right are both null and the Tree is NOT null.\\
This allows a data structure that can call upon itself to expand through the tree recursively.\\\\\\

\newpage
\fixindent{}
(b) Based on your recursive definition above, define (in code or mathematically) the function leaves(T)
that counts the number of leaves in a binary tree T. \moveToRight{4 marks}\\\\Solution in java:\\\\
\hspace*{150pt}public class Tree\textless  data\textgreater\\
\hspace*{150pt}\{\\
\hspace*{165pt} public data D;\\
\hspace*{165pt} Tree Left;\\
\hspace*{165pt} Tree Right;\\
\hspace*{150pt}\}\\\\







\end{document}
